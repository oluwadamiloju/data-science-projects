% Options for packages loaded elsewhere
\PassOptionsToPackage{unicode}{hyperref}
\PassOptionsToPackage{hyphens}{url}
%
\documentclass[
]{article}
\title{Factors that possibly influence suicide}
\author{Sarah Akinkunmi}
\date{11/20/2021}

\usepackage{amsmath,amssymb}
\usepackage{lmodern}
\usepackage{iftex}
\ifPDFTeX
  \usepackage[T1]{fontenc}
  \usepackage[utf8]{inputenc}
  \usepackage{textcomp} % provide euro and other symbols
\else % if luatex or xetex
  \usepackage{unicode-math}
  \defaultfontfeatures{Scale=MatchLowercase}
  \defaultfontfeatures[\rmfamily]{Ligatures=TeX,Scale=1}
\fi
% Use upquote if available, for straight quotes in verbatim environments
\IfFileExists{upquote.sty}{\usepackage{upquote}}{}
\IfFileExists{microtype.sty}{% use microtype if available
  \usepackage[]{microtype}
  \UseMicrotypeSet[protrusion]{basicmath} % disable protrusion for tt fonts
}{}
\makeatletter
\@ifundefined{KOMAClassName}{% if non-KOMA class
  \IfFileExists{parskip.sty}{%
    \usepackage{parskip}
  }{% else
    \setlength{\parindent}{0pt}
    \setlength{\parskip}{6pt plus 2pt minus 1pt}}
}{% if KOMA class
  \KOMAoptions{parskip=half}}
\makeatother
\usepackage{xcolor}
\IfFileExists{xurl.sty}{\usepackage{xurl}}{} % add URL line breaks if available
\IfFileExists{bookmark.sty}{\usepackage{bookmark}}{\usepackage{hyperref}}
\hypersetup{
  pdftitle={Factors that possibly influence suicide},
  pdfauthor={Sarah Akinkunmi},
  hidelinks,
  pdfcreator={LaTeX via pandoc}}
\urlstyle{same} % disable monospaced font for URLs
\usepackage[margin=1in]{geometry}
\usepackage{color}
\usepackage{fancyvrb}
\newcommand{\VerbBar}{|}
\newcommand{\VERB}{\Verb[commandchars=\\\{\}]}
\DefineVerbatimEnvironment{Highlighting}{Verbatim}{commandchars=\\\{\}}
% Add ',fontsize=\small' for more characters per line
\usepackage{framed}
\definecolor{shadecolor}{RGB}{248,248,248}
\newenvironment{Shaded}{\begin{snugshade}}{\end{snugshade}}
\newcommand{\AlertTok}[1]{\textcolor[rgb]{0.94,0.16,0.16}{#1}}
\newcommand{\AnnotationTok}[1]{\textcolor[rgb]{0.56,0.35,0.01}{\textbf{\textit{#1}}}}
\newcommand{\AttributeTok}[1]{\textcolor[rgb]{0.77,0.63,0.00}{#1}}
\newcommand{\BaseNTok}[1]{\textcolor[rgb]{0.00,0.00,0.81}{#1}}
\newcommand{\BuiltInTok}[1]{#1}
\newcommand{\CharTok}[1]{\textcolor[rgb]{0.31,0.60,0.02}{#1}}
\newcommand{\CommentTok}[1]{\textcolor[rgb]{0.56,0.35,0.01}{\textit{#1}}}
\newcommand{\CommentVarTok}[1]{\textcolor[rgb]{0.56,0.35,0.01}{\textbf{\textit{#1}}}}
\newcommand{\ConstantTok}[1]{\textcolor[rgb]{0.00,0.00,0.00}{#1}}
\newcommand{\ControlFlowTok}[1]{\textcolor[rgb]{0.13,0.29,0.53}{\textbf{#1}}}
\newcommand{\DataTypeTok}[1]{\textcolor[rgb]{0.13,0.29,0.53}{#1}}
\newcommand{\DecValTok}[1]{\textcolor[rgb]{0.00,0.00,0.81}{#1}}
\newcommand{\DocumentationTok}[1]{\textcolor[rgb]{0.56,0.35,0.01}{\textbf{\textit{#1}}}}
\newcommand{\ErrorTok}[1]{\textcolor[rgb]{0.64,0.00,0.00}{\textbf{#1}}}
\newcommand{\ExtensionTok}[1]{#1}
\newcommand{\FloatTok}[1]{\textcolor[rgb]{0.00,0.00,0.81}{#1}}
\newcommand{\FunctionTok}[1]{\textcolor[rgb]{0.00,0.00,0.00}{#1}}
\newcommand{\ImportTok}[1]{#1}
\newcommand{\InformationTok}[1]{\textcolor[rgb]{0.56,0.35,0.01}{\textbf{\textit{#1}}}}
\newcommand{\KeywordTok}[1]{\textcolor[rgb]{0.13,0.29,0.53}{\textbf{#1}}}
\newcommand{\NormalTok}[1]{#1}
\newcommand{\OperatorTok}[1]{\textcolor[rgb]{0.81,0.36,0.00}{\textbf{#1}}}
\newcommand{\OtherTok}[1]{\textcolor[rgb]{0.56,0.35,0.01}{#1}}
\newcommand{\PreprocessorTok}[1]{\textcolor[rgb]{0.56,0.35,0.01}{\textit{#1}}}
\newcommand{\RegionMarkerTok}[1]{#1}
\newcommand{\SpecialCharTok}[1]{\textcolor[rgb]{0.00,0.00,0.00}{#1}}
\newcommand{\SpecialStringTok}[1]{\textcolor[rgb]{0.31,0.60,0.02}{#1}}
\newcommand{\StringTok}[1]{\textcolor[rgb]{0.31,0.60,0.02}{#1}}
\newcommand{\VariableTok}[1]{\textcolor[rgb]{0.00,0.00,0.00}{#1}}
\newcommand{\VerbatimStringTok}[1]{\textcolor[rgb]{0.31,0.60,0.02}{#1}}
\newcommand{\WarningTok}[1]{\textcolor[rgb]{0.56,0.35,0.01}{\textbf{\textit{#1}}}}
\usepackage{graphicx}
\makeatletter
\def\maxwidth{\ifdim\Gin@nat@width>\linewidth\linewidth\else\Gin@nat@width\fi}
\def\maxheight{\ifdim\Gin@nat@height>\textheight\textheight\else\Gin@nat@height\fi}
\makeatother
% Scale images if necessary, so that they will not overflow the page
% margins by default, and it is still possible to overwrite the defaults
% using explicit options in \includegraphics[width, height, ...]{}
\setkeys{Gin}{width=\maxwidth,height=\maxheight,keepaspectratio}
% Set default figure placement to htbp
\makeatletter
\def\fps@figure{htbp}
\makeatother
\setlength{\emergencystretch}{3em} % prevent overfull lines
\providecommand{\tightlist}{%
  \setlength{\itemsep}{0pt}\setlength{\parskip}{0pt}}
\setcounter{secnumdepth}{-\maxdimen} % remove section numbering
\ifLuaTeX
  \usepackage{selnolig}  % disable illegal ligatures
\fi

\begin{document}
\maketitle

\hypertarget{my-analysis-on-factors-that-influence-suicide-dataset-in-austria-brazil-denmark-japan-norway-russia-federation-south-africa-and-the-united-states-of-america.-this-data-was-taken-from-kaggle.}{%
\subsection{My analysis on factors that influence suicide dataset in
Austria, Brazil, Denmark, Japan, Norway, Russia Federation, South Africa
and the United States of America. This data was taken from
Kaggle.}\label{my-analysis-on-factors-that-influence-suicide-dataset-in-austria-brazil-denmark-japan-norway-russia-federation-south-africa-and-the-united-states-of-america.-this-data-was-taken-from-kaggle.}}

First, I'll load the csv file:

\begin{Shaded}
\begin{Highlighting}[]
\FunctionTok{library}\NormalTok{(tidyverse)}
\end{Highlighting}
\end{Shaded}

\begin{verbatim}
## -- Attaching packages --------------------------------------- tidyverse 1.3.1 --
\end{verbatim}

\begin{verbatim}
## v ggplot2 3.3.5     v purrr   0.3.4
## v tibble  3.1.6     v dplyr   1.0.7
## v tidyr   1.1.4     v stringr 1.4.0
## v readr   2.1.0     v forcats 0.5.1
\end{verbatim}

\begin{verbatim}
## -- Conflicts ------------------------------------------ tidyverse_conflicts() --
## x dplyr::filter() masks stats::filter()
## x dplyr::lag()    masks stats::lag()
\end{verbatim}

\begin{Shaded}
\begin{Highlighting}[]
\NormalTok{file }\OtherTok{\textless{}{-}} \FunctionTok{read\_csv}\NormalTok{(}\StringTok{"C:/Sarah\textquotesingle{}s Semicolon files/data{-}science/project{-}1/master.csv"}\NormalTok{)}
\end{Highlighting}
\end{Shaded}

\begin{verbatim}
## Rows: 27820 Columns: 12
\end{verbatim}

\begin{verbatim}
## -- Column specification --------------------------------------------------------
## Delimiter: ","
## chr (5): country, sex, age, country-year, generation
## dbl (6): year, suicides_no, population, suicides/100k pop, HDI for year, gdp...
\end{verbatim}

\begin{verbatim}
## 
## i Use `spec()` to retrieve the full column specification for this data.
## i Specify the column types or set `show_col_types = FALSE` to quiet this message.
\end{verbatim}

\begin{Shaded}
\begin{Highlighting}[]
\NormalTok{file }\OtherTok{\textless{}{-}} \FunctionTok{rename}\NormalTok{(file, }\StringTok{\textasciigrave{}}\AttributeTok{suicide (\%)}\StringTok{\textasciigrave{}} \OtherTok{=} \StringTok{\textasciigrave{}}\AttributeTok{suicides/100k pop}\StringTok{\textasciigrave{}}\NormalTok{)}
\end{Highlighting}
\end{Shaded}

\hypertarget{then-ill-extract-each-country-into-its-own-dataframe}{%
\subsubsection{Then, I'll extract each country into it's own
dataframe:}\label{then-ill-extract-each-country-into-its-own-dataframe}}

For Austria:

\begin{Shaded}
\begin{Highlighting}[]
\NormalTok{au\_df }\OtherTok{\textless{}{-}}\NormalTok{ file}\SpecialCharTok{\%\textgreater{}\%}\NormalTok{.[}\DecValTok{1787}\SpecialCharTok{:}\DecValTok{2168}\NormalTok{, ]}
\end{Highlighting}
\end{Shaded}

\hypertarget{for-brazil}{%
\subsubsection{For Brazil:}\label{for-brazil}}

\begin{Shaded}
\begin{Highlighting}[]
\NormalTok{br\_df }\OtherTok{\textless{}{-}}\NormalTok{ file}\SpecialCharTok{\%\textgreater{}\%}\NormalTok{.[}\DecValTok{4173}\SpecialCharTok{:}\DecValTok{4544}\NormalTok{, ]}
\end{Highlighting}
\end{Shaded}

\hypertarget{for-denmark}{%
\subsubsection{For Denmark:}\label{for-denmark}}

\begin{Shaded}
\begin{Highlighting}[]
\NormalTok{denmark\_df }\OtherTok{\textless{}{-}}\NormalTok{ file}\SpecialCharTok{\%\textgreater{}\%}\NormalTok{.[}\DecValTok{7419}\SpecialCharTok{:}\DecValTok{7682}\NormalTok{, ]}
\end{Highlighting}
\end{Shaded}

\hypertarget{for-japan}{%
\subsubsection{For Japan:}\label{for-japan}}

\begin{Shaded}
\begin{Highlighting}[]
\NormalTok{jp\_df }\OtherTok{\textless{}{-}}\NormalTok{ file}\SpecialCharTok{\%\textgreater{}\%}\NormalTok{.[}\DecValTok{13365}\SpecialCharTok{:}\DecValTok{13736}\NormalTok{, ]}
\end{Highlighting}
\end{Shaded}

\hypertarget{for-norway}{%
\subsubsection{For Norway:}\label{for-norway}}

\begin{Shaded}
\begin{Highlighting}[]
\NormalTok{nw\_df }\OtherTok{\textless{}{-}}\NormalTok{ file}\SpecialCharTok{\%\textgreater{}\%}\NormalTok{.[}\DecValTok{17869}\SpecialCharTok{:}\DecValTok{18228}\NormalTok{, ]}
\end{Highlighting}
\end{Shaded}

\hypertarget{for-the-russia-federation}{%
\subsubsection{For the Russia
Federation:}\label{for-the-russia-federation}}

\begin{Shaded}
\begin{Highlighting}[]
\NormalTok{rs\_df }\OtherTok{\textless{}{-}}\NormalTok{ file}\SpecialCharTok{\%\textgreater{}\%}\NormalTok{.[}\DecValTok{20937}\SpecialCharTok{:}\DecValTok{21258}\NormalTok{, ]}
\end{Highlighting}
\end{Shaded}

\hypertarget{for-south-africa}{%
\subsubsection{For South Africa:}\label{for-south-africa}}

\begin{Shaded}
\begin{Highlighting}[]
\NormalTok{sa\_df }\OtherTok{\textless{}{-}}\NormalTok{ file}\SpecialCharTok{\%\textgreater{}\%}\NormalTok{.[}\DecValTok{23289}\SpecialCharTok{:}\DecValTok{23528}\NormalTok{, ]}
\end{Highlighting}
\end{Shaded}

\hypertarget{for-united-states-of-america}{%
\subsubsection{For United States of
America:}\label{for-united-states-of-america}}

\begin{Shaded}
\begin{Highlighting}[]
\NormalTok{us\_df }\OtherTok{\textless{}{-}}\NormalTok{ file}\SpecialCharTok{\%\textgreater{}\%}\NormalTok{.[}\DecValTok{26849}\SpecialCharTok{:}\DecValTok{27220}\NormalTok{, ]}
\end{Highlighting}
\end{Shaded}

\hypertarget{next-ill-perform-a-chi-square-test-between-the-age-and-generation-variables-in-all-the-countries.}{%
\subsubsection{Next, I'll perform a chi-square test between the age and
generation variables in all the
countries.}\label{next-ill-perform-a-chi-square-test-between-the-age-and-generation-variables-in-all-the-countries.}}

Null hypothesis (h\textsuperscript{0}): There is no relationship between
age and generation Alternative hypothesis (h\textsuperscript{a}): The
two variables are dependent on each other

For Austria:

\begin{Shaded}
\begin{Highlighting}[]
\FunctionTok{chisq.test}\NormalTok{(au\_df}\SpecialCharTok{$}\NormalTok{age, au\_df}\SpecialCharTok{$}\NormalTok{generation)}
\end{Highlighting}
\end{Shaded}

\begin{verbatim}
## Warning in chisq.test(au_df$age, au_df$generation): Chi-squared approximation
## may be incorrect
\end{verbatim}

\begin{verbatim}
## 
##  Pearson's Chi-squared test
## 
## data:  au_df$age and au_df$generation
## X-squared = 586.97, df = 25, p-value < 2.2e-16
\end{verbatim}

\hypertarget{since-the-p-value-is-less-than-0.05-2.2e-16-we-will-reject-the-null-hypothesis.-we-can-conclude-that-the-age-and-generation-depend-on-each-other}{%
\subsubsection{\texorpdfstring{Since the p-value is less than 0.05
(2.2e\textsuperscript{-16}), we will reject the null hypothesis. We can
conclude that the age and generation depend on each
other}{Since the p-value is less than 0.05 (2.2e-16), we will reject the null hypothesis. We can conclude that the age and generation depend on each other}}\label{since-the-p-value-is-less-than-0.05-2.2e-16-we-will-reject-the-null-hypothesis.-we-can-conclude-that-the-age-and-generation-depend-on-each-other}}

\hypertarget{for-brazil-1}{%
\subsection{For Brazil:}\label{for-brazil-1}}

\begin{Shaded}
\begin{Highlighting}[]
\FunctionTok{chisq.test}\NormalTok{(br\_df}\SpecialCharTok{$}\NormalTok{age, br\_df}\SpecialCharTok{$}\NormalTok{generation)}
\end{Highlighting}
\end{Shaded}

\begin{verbatim}
## Warning in chisq.test(br_df$age, br_df$generation): Chi-squared approximation
## may be incorrect
\end{verbatim}

\begin{verbatim}
## 
##  Pearson's Chi-squared test
## 
## data:  br_df$age and br_df$generation
## X-squared = 588.2, df = 25, p-value < 2.2e-16
\end{verbatim}

\hypertarget{since-the-p-value-is-less-than-0.05-2.2e-16-we-will-reject-the-null-hypothesis.-we-can-conclude-that-the-age-and-generation-depend-on-each-other-1}{%
\subsubsection{\texorpdfstring{Since the p-value is less than 0.05
(2.2e\textsuperscript{-16}), we will reject the null hypothesis. We can
conclude that the age and generation depend on each
other}{Since the p-value is less than 0.05 (2.2e-16), we will reject the null hypothesis. We can conclude that the age and generation depend on each other}}\label{since-the-p-value-is-less-than-0.05-2.2e-16-we-will-reject-the-null-hypothesis.-we-can-conclude-that-the-age-and-generation-depend-on-each-other-1}}

\hypertarget{for-denmark-1}{%
\subsection{For Denmark:}\label{for-denmark-1}}

\begin{Shaded}
\begin{Highlighting}[]
\FunctionTok{chisq.test}\NormalTok{(denmark\_df}\SpecialCharTok{$}\NormalTok{age, denmark\_df}\SpecialCharTok{$}\NormalTok{generation)}
\end{Highlighting}
\end{Shaded}

\begin{verbatim}
## Warning in chisq.test(denmark_df$age, denmark_df$generation): Chi-squared
## approximation may be incorrect
\end{verbatim}

\begin{verbatim}
## 
##  Pearson's Chi-squared test
## 
## data:  denmark_df$age and denmark_df$generation
## X-squared = 568.04, df = 25, p-value < 2.2e-16
\end{verbatim}

\hypertarget{since-the-p-value-is-less-than-0.05-2.2e-16-we-will-reject-the-null-hypothesis.-we-can-conclude-that-the-age-and-generation-depend-on-each-other-2}{%
\subsubsection{\texorpdfstring{Since the p-value is less than 0.05
(2.2e\textsuperscript{-16}), we will reject the null hypothesis. We can
conclude that the age and generation depend on each
other}{Since the p-value is less than 0.05 (2.2e-16), we will reject the null hypothesis. We can conclude that the age and generation depend on each other}}\label{since-the-p-value-is-less-than-0.05-2.2e-16-we-will-reject-the-null-hypothesis.-we-can-conclude-that-the-age-and-generation-depend-on-each-other-2}}

\hypertarget{for-japan-1}{%
\subsection{For Japan:}\label{for-japan-1}}

\begin{Shaded}
\begin{Highlighting}[]
\FunctionTok{chisq.test}\NormalTok{(jp\_df}\SpecialCharTok{$}\NormalTok{age, jp\_df}\SpecialCharTok{$}\NormalTok{generation)}
\end{Highlighting}
\end{Shaded}

\begin{verbatim}
## Warning in chisq.test(jp_df$age, jp_df$generation): Chi-squared approximation
## may be incorrect
\end{verbatim}

\begin{verbatim}
## 
##  Pearson's Chi-squared test
## 
## data:  jp_df$age and jp_df$generation
## X-squared = 588.2, df = 25, p-value < 2.2e-16
\end{verbatim}

\hypertarget{since-the-p-value-is-less-than-0.05-2.2e-16-we-will-reject-the-null-hypothesis.-we-can-conclude-that-the-age-and-generation-depend-on-each-other-3}{%
\subsubsection{\texorpdfstring{Since the p-value is less than 0.05
(2.2e\textsuperscript{-16}), we will reject the null hypothesis. We can
conclude that the age and generation depend on each
other}{Since the p-value is less than 0.05 (2.2e-16), we will reject the null hypothesis. We can conclude that the age and generation depend on each other}}\label{since-the-p-value-is-less-than-0.05-2.2e-16-we-will-reject-the-null-hypothesis.-we-can-conclude-that-the-age-and-generation-depend-on-each-other-3}}

\hypertarget{for-norway-1}{%
\subsection{For Norway:}\label{for-norway-1}}

\begin{Shaded}
\begin{Highlighting}[]
\FunctionTok{chisq.test}\NormalTok{(nw\_df}\SpecialCharTok{$}\NormalTok{age, nw\_df}\SpecialCharTok{$}\NormalTok{generation)}
\end{Highlighting}
\end{Shaded}

\begin{verbatim}
## Warning in chisq.test(nw_df$age, nw_df$generation): Chi-squared approximation
## may be incorrect
\end{verbatim}

\begin{verbatim}
## 
##  Pearson's Chi-squared test
## 
## data:  nw_df$age and nw_df$generation
## X-squared = 586.34, df = 25, p-value < 2.2e-16
\end{verbatim}

\hypertarget{since-the-p-value-is-less-than-0.05-2.2e-16-we-will-reject-the-null-hypothesis.-we-can-conclude-that-the-age-and-generation-depend-on-each-other-4}{%
\subsubsection{\texorpdfstring{Since the p-value is less than 0.05
(2.2e\textsuperscript{-16}), we will reject the null hypothesis. We can
conclude that the age and generation depend on each
other}{Since the p-value is less than 0.05 (2.2e-16), we will reject the null hypothesis. We can conclude that the age and generation depend on each other}}\label{since-the-p-value-is-less-than-0.05-2.2e-16-we-will-reject-the-null-hypothesis.-we-can-conclude-that-the-age-and-generation-depend-on-each-other-4}}

\hypertarget{for-russia}{%
\subsection{For Russia:}\label{for-russia}}

\begin{Shaded}
\begin{Highlighting}[]
\FunctionTok{chisq.test}\NormalTok{(rs\_df}\SpecialCharTok{$}\NormalTok{age, rs\_df}\SpecialCharTok{$}\NormalTok{generation)}
\end{Highlighting}
\end{Shaded}

\begin{verbatim}
## Warning in chisq.test(rs_df$age, rs_df$generation): Chi-squared approximation
## may be incorrect
\end{verbatim}

\begin{verbatim}
## 
##  Pearson's Chi-squared test
## 
## data:  rs_df$age and rs_df$generation
## X-squared = 590.35, df = 25, p-value < 2.2e-16
\end{verbatim}

\hypertarget{since-the-p-value-is-less-than-0.05-2.2e-16-we-will-reject-the-null-hypothesis.-we-can-conclude-that-the-age-and-generation-depend-on-each-other-5}{%
\subsubsection{\texorpdfstring{Since the p-value is less than 0.05
(2.2e\textsuperscript{-16}), we will reject the null hypothesis. We can
conclude that the age and generation depend on each
other}{Since the p-value is less than 0.05 (2.2e-16), we will reject the null hypothesis. We can conclude that the age and generation depend on each other}}\label{since-the-p-value-is-less-than-0.05-2.2e-16-we-will-reject-the-null-hypothesis.-we-can-conclude-that-the-age-and-generation-depend-on-each-other-5}}

\hypertarget{for-south-africa-1}{%
\subsection{For South Africa:}\label{for-south-africa-1}}

\begin{Shaded}
\begin{Highlighting}[]
\FunctionTok{chisq.test}\NormalTok{(sa\_df}\SpecialCharTok{$}\NormalTok{age, sa\_df}\SpecialCharTok{$}\NormalTok{generation)}
\end{Highlighting}
\end{Shaded}

\begin{verbatim}
## Warning in chisq.test(sa_df$age, sa_df$generation): Chi-squared approximation
## may be incorrect
\end{verbatim}

\begin{verbatim}
## 
##  Pearson's Chi-squared test
## 
## data:  sa_df$age and sa_df$generation
## X-squared = 523.19, df = 25, p-value < 2.2e-16
\end{verbatim}

\hypertarget{since-the-p-value-is-less-than-0.05-2.2e-16-we-will-reject-the-null-hypothesis.-we-can-conclude-that-the-age-and-generation-depend-on-each-other-6}{%
\subsubsection{\texorpdfstring{Since the p-value is less than 0.05
(2.2e\textsuperscript{-16}), we will reject the null hypothesis. We can
conclude that the age and generation depend on each
other}{Since the p-value is less than 0.05 (2.2e-16), we will reject the null hypothesis. We can conclude that the age and generation depend on each other}}\label{since-the-p-value-is-less-than-0.05-2.2e-16-we-will-reject-the-null-hypothesis.-we-can-conclude-that-the-age-and-generation-depend-on-each-other-6}}

\hypertarget{for-united-states-of-america-1}{%
\subsection{For United States of
America:}\label{for-united-states-of-america-1}}

\begin{Shaded}
\begin{Highlighting}[]
\FunctionTok{chisq.test}\NormalTok{(us\_df}\SpecialCharTok{$}\NormalTok{age, us\_df}\SpecialCharTok{$}\NormalTok{generation)}
\end{Highlighting}
\end{Shaded}

\begin{verbatim}
## Warning in chisq.test(us_df$age, us_df$generation): Chi-squared approximation
## may be incorrect
\end{verbatim}

\begin{verbatim}
## 
##  Pearson's Chi-squared test
## 
## data:  us_df$age and us_df$generation
## X-squared = 588.2, df = 25, p-value < 2.2e-16
\end{verbatim}

\hypertarget{since-the-p-value-is-less-than-0.05-2.2e-16-we-will-reject-the-null-hypothesis.-we-can-conclude-that-the-age-and-generation-depend-on-each-other-7}{%
\subsubsection{\texorpdfstring{Since the p-value is less than 0.05
(2.2e\textsuperscript{-16}), we will reject the null hypothesis. We can
conclude that the age and generation depend on each
other}{Since the p-value is less than 0.05 (2.2e-16), we will reject the null hypothesis. We can conclude that the age and generation depend on each other}}\label{since-the-p-value-is-less-than-0.05-2.2e-16-we-will-reject-the-null-hypothesis.-we-can-conclude-that-the-age-and-generation-depend-on-each-other-7}}

\hypertarget{our-chi-square-analysis-of-age-and-generation-are-the-same-in-all-countries-so-its-safe-to-conclude-that-we-can-use-either-variable-but-not-both-for-other-analysis}{%
\subsubsection{Our chi-square analysis of age and generation are the
same in all countries, so it's safe to conclude that we can use either
variable but not both for other
analysis}\label{our-chi-square-analysis-of-age-and-generation-are-the-same-in-all-countries-so-its-safe-to-conclude-that-we-can-use-either-variable-but-not-both-for-other-analysis}}

\hypertarget{next-ill-perform-a-chi-square-test-between-the-age-and-sex-variables-in-all-the-countries.}{%
\subsubsection{Next, I'll perform a chi-square test between the age and
sex variables in all the
countries.}\label{next-ill-perform-a-chi-square-test-between-the-age-and-sex-variables-in-all-the-countries.}}

Null hypothesis (h\textsuperscript{0}): There is no relationship between
age and generation Alternative hypothesis (h\textsuperscript{A}): The
two variables are dependent on each other

\hypertarget{for-austria}{%
\subsection{For Austria:}\label{for-austria}}

\begin{Shaded}
\begin{Highlighting}[]
\FunctionTok{chisq.test}\NormalTok{(au\_df}\SpecialCharTok{$}\NormalTok{age, au\_df}\SpecialCharTok{$}\NormalTok{sex)}
\end{Highlighting}
\end{Shaded}

\begin{verbatim}
## 
##  Pearson's Chi-squared test
## 
## data:  au_df$age and au_df$sex
## X-squared = 0, df = 5, p-value = 1
\end{verbatim}

Since our p-value is greater than 0.05 (1), we fail to reject the null
hypothesis and we conclude that both variables are independent

\hypertarget{for-brazil-2}{%
\subsection{For Brazil:}\label{for-brazil-2}}

\begin{Shaded}
\begin{Highlighting}[]
\FunctionTok{chisq.test}\NormalTok{(br\_df}\SpecialCharTok{$}\NormalTok{age, br\_df}\SpecialCharTok{$}\NormalTok{sex)}
\end{Highlighting}
\end{Shaded}

\begin{verbatim}
## 
##  Pearson's Chi-squared test
## 
## data:  br_df$age and br_df$sex
## X-squared = 0, df = 5, p-value = 1
\end{verbatim}

Since our p-value is greater than 0.05 (1), we fail to reject the null
hypothesis and we conclude that both variables are independent

\hypertarget{for-denmark-2}{%
\subsection{For Denmark:}\label{for-denmark-2}}

\begin{Shaded}
\begin{Highlighting}[]
\FunctionTok{chisq.test}\NormalTok{(denmark\_df}\SpecialCharTok{$}\NormalTok{age, denmark\_df}\SpecialCharTok{$}\NormalTok{sex)}
\end{Highlighting}
\end{Shaded}

\begin{verbatim}
## 
##  Pearson's Chi-squared test
## 
## data:  denmark_df$age and denmark_df$sex
## X-squared = 0, df = 5, p-value = 1
\end{verbatim}

Since our p-value is greater than 0.05 (1), we fail to reject the null
hypothesis and we conclude that both variables are independent

\hypertarget{for-japan-2}{%
\subsection{For Japan:}\label{for-japan-2}}

\begin{Shaded}
\begin{Highlighting}[]
\FunctionTok{chisq.test}\NormalTok{(jp\_df}\SpecialCharTok{$}\NormalTok{age, jp\_df}\SpecialCharTok{$}\NormalTok{sex)}
\end{Highlighting}
\end{Shaded}

\begin{verbatim}
## 
##  Pearson's Chi-squared test
## 
## data:  jp_df$age and jp_df$sex
## X-squared = 0, df = 5, p-value = 1
\end{verbatim}

Since our p-value is greater than 0.05 (1), we fail to reject the null
hypothesis and we conclude that both variables are independent

\hypertarget{for-norway-2}{%
\subsection{For Norway:}\label{for-norway-2}}

\begin{Shaded}
\begin{Highlighting}[]
\FunctionTok{chisq.test}\NormalTok{(nw\_df}\SpecialCharTok{$}\NormalTok{age, nw\_df}\SpecialCharTok{$}\NormalTok{sex)}
\end{Highlighting}
\end{Shaded}

\begin{verbatim}
## 
##  Pearson's Chi-squared test
## 
## data:  nw_df$age and nw_df$sex
## X-squared = 0, df = 5, p-value = 1
\end{verbatim}

Since our p-value is greater than 0.05 (1), we fail to reject the null
hypothesis and we conclude that both variables are independent

\hypertarget{for-russia-1}{%
\subsection{For Russia:}\label{for-russia-1}}

\begin{Shaded}
\begin{Highlighting}[]
\FunctionTok{chisq.test}\NormalTok{(rs\_df}\SpecialCharTok{$}\NormalTok{age, rs\_df}\SpecialCharTok{$}\NormalTok{sex)}
\end{Highlighting}
\end{Shaded}

\begin{verbatim}
## 
##  Pearson's Chi-squared test
## 
## data:  rs_df$age and rs_df$sex
## X-squared = 0, df = 5, p-value = 1
\end{verbatim}

Since our p-value is greater than 0.05 (1), we fail to reject the null
hypothesis and we conclude that both variables are independent

\hypertarget{for-south-africa-2}{%
\subsection{For South Africa:}\label{for-south-africa-2}}

\begin{Shaded}
\begin{Highlighting}[]
\FunctionTok{chisq.test}\NormalTok{(sa\_df}\SpecialCharTok{$}\NormalTok{age, sa\_df}\SpecialCharTok{$}\NormalTok{sex)}
\end{Highlighting}
\end{Shaded}

\begin{verbatim}
## 
##  Pearson's Chi-squared test
## 
## data:  sa_df$age and sa_df$sex
## X-squared = 0, df = 5, p-value = 1
\end{verbatim}

Since our p-value is greater than 0.05 (1), we fail to reject the null
hypothesis and we conclude that both variables are independent

\hypertarget{for-united-states-of-america-2}{%
\subsection{For United States of
America:}\label{for-united-states-of-america-2}}

\begin{Shaded}
\begin{Highlighting}[]
\FunctionTok{chisq.test}\NormalTok{(us\_df}\SpecialCharTok{$}\NormalTok{age, us\_df}\SpecialCharTok{$}\NormalTok{sex)}
\end{Highlighting}
\end{Shaded}

\begin{verbatim}
## 
##  Pearson's Chi-squared test
## 
## data:  us_df$age and us_df$sex
## X-squared = 0, df = 5, p-value = 1
\end{verbatim}

Since our p-value is greater than 0.05 (1), we fail to reject the null
hypothesis and we conclude that both variables are independent

Therefore, we can safely use both categorical variables in any other
analysis

\hypertarget{anova-tests}{%
\section{ANOVA tests}\label{anova-tests}}

One-way ANOVA tests between Suicide \% and age For Austria:

\begin{Shaded}
\begin{Highlighting}[]
\NormalTok{one\_way\_au }\OtherTok{\textless{}{-}} \FunctionTok{aov}\NormalTok{(}\StringTok{\textasciigrave{}}\AttributeTok{suicide (\%)}\StringTok{\textasciigrave{}}\SpecialCharTok{\textasciitilde{}}\NormalTok{ age, }\AttributeTok{data =}\NormalTok{ au\_df)}
\FunctionTok{summary}\NormalTok{(one\_way\_au)}
\end{Highlighting}
\end{Shaded}

\begin{verbatim}
##              Df Sum Sq Mean Sq F value Pr(>F)    
## age           5 132286   26457    69.8 <2e-16 ***
## Residuals   376 142524     379                   
## ---
## Signif. codes:  0 '***' 0.001 '**' 0.01 '*' 0.05 '.' 0.1 ' ' 1
\end{verbatim}

The model summary first lists the independent variables being tested in
the model (in this case we have only one, `fertilizer') and the model
residuals (`Residual'). All of the variation that is not explained by
the independent variables is called residual variance.

The rest of the values in the output table describe the independent
variable and the residuals:

The Df column displays the degrees of freedom for the independent
variable (the number of levels in the variable minus 1), and the degrees
of freedom for the residuals (the total number of observations minus one
and minus the number of levels in the independent variables). The Sum Sq
column displays the sum of squares (a.k.a. the total variation between
the group means and the overall mean). The Mean Sq column is the mean of
the sum of squares, calculated by dividing the sum of squares by the
degrees of freedom for each parameter. The F-value column is the test
statistic from the F test. This is the mean square of each independent
variable divided by the mean square of the residuals. The larger the F
value, the more likely it is that the variation caused by the
independent variable is real and not due to chance. The
Pr(\textgreater F) column is the p-value of the F-statistic. This shows
how likely it is that the F-value calculated from the test would have
occurred if the null hypothesis of no difference among group means were
true. The p-value of the fertilizer variable is low (p \textless{}
0.001), so it appears that the type of fertilizer used has a real impact
on the final crop yield.

\end{document}
